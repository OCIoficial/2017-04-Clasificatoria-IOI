\documentclass{oci}
\usepackage[utf8]{inputenc}
\usepackage{lipsum}

\title{Dedos Rápidos}
\codename{dedos}

\begin{document}
\begin{problemDescription}
Un famoso videojuego de ritmo musical consiste en hacer al jugador tocar
canciones de rock con una guitarra eléctrica simplificada.
La guitarra simplificada consiste en $B$ botones dispuestos en una línea, los
cuales representan los trastes de una guitarra eléctrica.
El objetivo es acertar a tocar la mayor cantidad de notas de la canción que
sea posible.

Una canción está formada por $T$ tiempos, en los cuales pueden ocurrir notas
musicales o no.
Cada nota musical en la canción ocurre en un tiempo $t_i$, dura exactamente
un tiempo y está asociada a un botón $b_j$ (traste).
Pueden haber varias notas al mismo tiempo, y pueden haber varias notas
consecutivas asociadas al mismo botón, pero no puede haber dos notas que
ocurren al mismo tiempo para el mismo botón.

Coni es fanática de este videojuego, y para cada canción quiere saber el máximo
puntaje que puede alcanzar usando $D$ dedos.
Su problema es que, dependiendo del ritmo de la canción, puede que no sea
capaz de mover los dedos suficientemente rápido.
Cada dedo puede moverse de un botón a otro en $C$ tiempos, durante los cuales
ese dedo no puede presionar ningún botón.
Es posible para Coni cruzar sus dedos mientras juega, y también es posible
tener dos dedos sobre el mismo botón.

\end{problemDescription}

Para resolver este problema, te pedimos implementar la función
\begin{verbatim}
int max_puntaje(vector< vector< bool > > cancion, int d, int c)
\end{verbatim}
que describimos a continuación:

\begin{inputDescription}
\begin{itemize}
 \item \verb#cancion#: un vector de dos dimensiones de tamaño $T \times B$,
 donde \verb#cancion[i]# indica las notas de la canción para el tiempo
 $i$-ésimo, en particular, \verb#cancion[i][j]# es verdadero si en el tiempo
 $i$-ésimo hay una nota asociada al botón $j$-ésimo, y falso si no.
 \item \verb#d#: es el número de dedos que usará Coni.
 \item \verb#c#: es número de tiempos que toma cambiar un dedo de un botón
 a otro.
\end{itemize}
\end{inputDescription}

\begin{outputDescription}
La salida del problema debe ser un entero $M$, el máximo número de notas que
Coni puede acertar en \verb#cancion# usando $d$ dedos, considerando que tras un
cambio de botón, un dedo no puede ser usado durante \verb#c# tiempos.
\end{outputDescription}

\begin{scoreDescription}
  \score{10} Cada caso tendrá $d = 1$ y $c = 0$. Además, $T = 5$ y $B \le 100$.
  \score{25} Cada caso tendrá $d = 2$ y $c = 0$. Además, $T = 5$ y $B \le 100$.
  \score{30} Cada caso tendrá $d = 1$ y $c = 1$. Además, $T = 5$ y $B \le 1000$.
  \score{35} Cada caso tendrá $2 \le d \le 3$ y $1 \le c \le 5$. Además, $T = 5$ y $B \le 1000$.
\end{scoreDescription}

\begin{sampleDescription}
\sampleIO{sample-1}
\sampleIO{sample-2}
\end{sampleDescription}

\end{document}
