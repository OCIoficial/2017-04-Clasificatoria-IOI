\documentclass{oci}
\usepackage[utf8]{inputenc}
\usepackage{lipsum}
\usepackage{diagbox}

\title{Dedos Rápidos}
\codename{dedos}

\begin{document}
\begin{problemDescription}
  Dedos rápidos es un famoso videojuego de ritmo donde los jugadores simulan
  estar en una banda de rock tocando una guitarra eléctrica simplificada.
  Esta guitarra consiste en un mástil con 5 botones representando los trastes de
  la guitarra.
  A medida que la canción progresa distintas marcas son mostradas en la pantalla
  indicando que botones debe presionar el jugador para tocar las notas de la canción.
  El objetivo es acertar la mayor cantidad de notas posibles.

  Para simplificar la descripción, diremos que una canción está compuesta por $T$
  \emph{beats}, en los cuales pueden o no ocurrir notas musicales.
  Cada beat se describe con 5 enteros que pueden ser 0 o 1 indicando en qué
  posiciones hay notas.
  Un 1 indica que hay una nota en esa posición y para tocarla hay que presionar
  el botón en la misma posición.
  Un 0 indica la ausencia de una nota en una posición.
  Pueden haber varias notas en un mismo beat y pueden haber varias
  notas consecutivas asociadas al mismo botón.

  La figura de ejemplo que se muestra a continuación contiene la descripción de
  6 beats consecutivos.
  En el primer beat hay notas en las posiciones 3 y 5.
  En el siguiente beat hay una nota solo en la posición 3.
  Posteriormente, en el tercer beat no hay ninguna nota.
  En el beat 4 hay una en la posición 1 y otra en la posición 3.
  En el beat 5 no hay ninguna nota.
  Finalmente, el último beat contiene una sola nota en la posición 4.

  \begin{center}
    \begin{tabular}{r|ccccc}
        & 1 & 2 & 3 & 4 & 5 \\
      \hline
      1 & 0 & 0 & 1 & 0 & 1 \\
      2 & 0 & 0 & 1 & 0 & 0 \\
      3 & 0 & 0 & 0 & 0 & 0 \\
      4 & 1 & 0 & 1 & 0 & 0 \\
      5 & 0 & 0 & 0 & 0 & 0 \\
      6 & 0 & 0 & 0 & 1 & 0 \\
    \end{tabular}
  \end{center}

  Coni es fanática de este videojuego y esta interesada en estudiar como
  maximizar su puntaje al jugar.
  Inicialmente ella puede escoger en qué posiciones pondrá sus dedos y durante
  la canción deberá moverlos a otras posiciones para alcanzar a tocar otras notas.
  Al inicio de cada beat Coni puede decidir si presionar un botón, mantener un
  dedo en una posición sin hacer nada con él o comenzar a mover uno.
  Si Coni decide presionar un botón deberá mantenerlo presionado durante todo el
  beat y al inicio del siguiente beat el dedo se mantendrá en la misma posición.
  Si Coni decide comenzar a mover un dedo este se mantendrá inhabilitado durante
  $C$ beats, valor que depende de la velocidad y ritmo de la canción.
  Durante estos $C$ beats no podrá presionar ningún botón con el dedo, pero este
  puede ser movido a cualquier otra posición.
  Por ejemplo si $C=1$ y Coni decide empezar a mover un dedo en el beat 1 este
  se mantendrá inhabilitado durante el beat 1 y en el beat 2 ya estará
  habilitado para presionar un botón en cualquier posición.
  Coni posee una técnica extraordinaria que le permite mover cada dedo
  de forma totalmente independiente, cruzar sus dedos mientras juega y mantener
  más de un dedo sobre la misma posición.

  Para cada canción Coni desea saber cuál es la cantidad máxima de notas que
  puede alcanzar a tocar dependiendo de la cantidad de dedos que utilice.
  Su problema es que, dependiendo del ritmo de la canción, puede que no sea
  capaz de mover los dedos lo suficientemente rápido.
  Por ejemplo, considera la descripción de la figura anterior y que Coni juega
  con 2 dedos y tarda $C=2$ beats en mover los dedos de posición.
  Ella podría escoger comenzar con sus dedos en las posiciones 3 y 5, y en el
  primer beat decidir presionar los botones en estas mismas posiciones, para así
  tocar las dos notas presentes en este beat.
  En el siguiente beat puede decidir presionar el botón en la posición 3
  tocando la única nota de este beat y a la vez comenzar a mover el dedo que
  tenía en la posición 5.
  En el tercer beat Coni aún mantiene un dedo en movimiento y no puede hacer nada
  más con él.
  El otro dedo se encuentra en la posición 3 y podría decidir mantenerlo en esta
  posición sin hacer nada.
  En el beat 4 el dedo que había comenzado a mover ya podría haber alcanzado
  la posición 1, además como el otro dedo se mantuvo en la posición 3 Coni puede
  tocar las dos notas de este beat.
  En el beat 5 Coni tiene sus dedos en las posiciones 1 y 3, pero sin importar
  lo que decida hacer no alcanzará a mover sus dedos para tocar la nota en la
  posición 6.
\end{problemDescription}
\begin{inputDescription}
  La primera línea de la entrada contiene tres enteros $T$, $D$ y $C$
  correspondientes respectivamente a la cantidad de beats en la canción, la
  cantidad de dedos con los que juega Coni y la cantidad de beats que Coni
  tarda en mover un dedo de una posición a otra.
  Las siguientes $T$ líneas contienen la descripción de cada beat.
  Cada línea contiene 5 enteros que pueden ser 0 o 1.
  Un 0 en la posición $i$-ésima indica que no hay una nota en esa posición.
  Un 1 indica la presencia de una nota en la posición.
\end{inputDescription}

\begin{outputDescription}
  La salida debe corresponder a un único entero $M$ correspondiente al número
  máximo de notas que Coni puede alcanzar a tocar.
\end{outputDescription}

\begin{scoreDescription}
  \score{10} Se probarán varios casos donde $D = 1$, $C = 0$ y además $0 < T \le 100$.
  \score{25} Se probarán varios casos donde $D = 2$ y $C = 0$ y además $0 < T \le 100$.
  \score{30} Se probarán varios casos donde $D = 1$ y $C = 1$ y además $0 < T \le 1000$.
  \score{35} Se probarán varios casos donde $2 \le D \le 3$ y $1 \le C \le 5$ y
  además $0 < T \le 1000$.
\end{scoreDescription}

\begin{sampleDescription}
\sampleIO{sample-1}
\sampleIO{sample-2}
\end{sampleDescription}

\end{document}
