\documentclass{oci}
\usepackage[utf8]{inputenc}
\usepackage{lipsum}
\usepackage{diagbox}

\title{Dedos Rápidos}
\codename{dedos}

\begin{document}
\begin{problemDescription}
  Dedos rápidos es un famoso videojuego de ritmo donde los jugadores simulan
  estar en una banda de rock tocando una guitarra eléctrica simplificada.
  La guitarra simplificada consiste en un mástil con 5 botones representando los
  trastes de la guitarra.
  A medida que la canción progresa distintas marcas son mostradas en la pantalla
  indicando que botones debe presionar el jugador para tocar las notas de la canción.
  El objetivo es acertar la mayor cantidad de notas posibles.

  Para simplificar la descripción diremos que una canción está compuesta por $T$
  tiempos, en los cuales pueden o no ocurrir notas musicales.
  Cada tiempo se describe con 5 enteros que pueden ser 0 o 1 indicando en qué
  posiciones hay notas.
  Un 1 indica que hay una nota en esa posición y para tocar esa nota hay que
  presionar el botón en la misma posición.
  Un 0 indica la ausencia de una nota en una posición.
  Pueden haber varias notas en un mismo tiempo y pueden haber varias
  notas consecutivas asociadas al mismo botón.

  La figura que se muestra a continuación contiene la descripción de 6 tiempos
  consecutivos de ejemplo.
  En el primer tiempo hay notas en la posiciones 3 y 5.
  En el siguiente tiempo hay una nota solo en la posición 3.
  Posteriormente en el tiempo 3 no hay ninguna nota.
  En el tiempo 4 hay una en la posición 3 y otra en la posición 1.
  Luego en el tiempo 5 no hay ninguna nota.
  Finalmente, el último tiempo tiene una sola nota en la posición 4.

  \begin{center}
    \begin{tabular}{r|ccccc}
        & 1 & 2 & 3 & 4 & 5 \\
      \hline
      1 & 0 & 0 & 1 & 0 & 1 \\
      2 & 0 & 0 & 1 & 0 & 0 \\
      3 & 0 & 0 & 0 & 0 & 0 \\
      4 & 1 & 0 & 1 & 0 & 0 \\
      5 & 0 & 0 & 0 & 0 & 0 \\
      6 & 0 & 0 & 0 & 1 & 0 \\
    \end{tabular}
  \end{center}

  Coni es fanática de este videojuego y esta interesada en estudiar como
  maximizar su puntaje al jugar.
  Ella puede escoger en qué posiciones pondrá inicialmente sus dedos y mover un
  dedo de una posición a cualquier otra le toma $C$ tiempos, valor que depende
  de la velocidad y ritmo de la canción.
  Por otro lado presionar un botón teniendo un dedo ya en la posición no toma
  nada de tiempo.
  En cada tiempo Coni puede decidir si presionar un botón con un dedo
  manteniéndolo en esa posición, mantener un dedo en una posición sin hacer nada
  o comenzar a mover un dedo.
  Durante el tiempo que un dedo se encuentre en movimiento no puede presionar
  ningún botón.
  Coni posee una técnica extraordinaria que le permite mover cada dedo
  de forma totalmente independiente, cruzar sus dedos mientras juega y mantener
  más de un dedo sobre la misma posición.

  Para cada canción Coni desea saber cuál es la cantidad máxima de notas que
  puede alcanzar a tocar dependiendo de la cantidad de dedos que utilice.
  Su problema es que, dependiendo del ritmo de la canción, puede que no sea
  capaz de mover los dedos lo suficientemente rápido.
  Por ejemplo, considera la descripción de la figura anterior y que Coni juega
  con 2 dedos y tarda $C=2$ tiempos en mover los dedos de posición.
  Ella podría escoger comenzar con sus dedos en las posiciones 3 y 5, y en el
  primer tiempo decir presionar los botones en estas mismas posiciones, para así
  tocar las dos notas presentes en este tiempo.
  En el siguiente tiempo puede decidir presionar el botón en la posición 3
  tocando la única nota de este tiempo y comenzar a mover el dedo que
  tenía en la posición 5.
  En el tiempo 3 Coni aún mantiene un dedo en movimiento y no puede hacer nada
  más con él.
  El otro dedo se encuentra en la posición 3 y podría decidir mantenerlo en esta
  posición.
  En el tiempo 4 el dedo que había comenzado a mover ya podría haber alcanzado
  la posición 1, además como el otro dedo se mantuvo en la posición 3 Coni puede
  tocar las dos notas de este tiempo.
  En el tiempo 5 Coni tiene sus dedos en las posiciones 1 y 3, pero sin importar
  lo que decida hacer no alcanzará a mover sus dedos para tocar la nota en la
  posición 6.
\end{problemDescription}
\begin{inputDescription}
  La primera línea de la entrada contiene tres enteros $T$, $D$ y $C$
  correspondientes respectivamente a la cantidad de tiempos en la canción, la
  cantidad de dedos con los que juega Coni y la cantidad de tiempos que Coni
  tarda en mover un dedo de una posición a otra.
  Las siguientes $T$ líneas contienen la descripción de cada tiempo.
  Cada línea contiene 5 enteros que pueden ser 0 o 1.
  Un 0 en la posición $i$-ésima indica que no hay una nota en esa posición.
  Un 1 indica la presencia de una nota en la posición.
\end{inputDescription}

\begin{outputDescription}
  La salida debe corresponder a un único entero $M$ correspondiente al número
  máximo de notas que Coni puede alcanzar a tocar.
\end{outputDescription}

\begin{scoreDescription}
  \score{10} Se probarán varios casos donde $D = 1$, $C = 0$ y además $0 < T \le 100$.
  \score{25} Se probarán varios casos donde $D = 2$ y $C = 0$ y además $0 < T \le 100$.
  \score{30} Se probarán varios casos donde $D = 1$ y $C = 1$ y además $0 < T \le 1000$.
  \score{35} Se probarán varios casos donde $2 \le D \le 3$ y $1 \le C \le 5$ y
  además $0 < T \le 1000$.
\end{scoreDescription}

\begin{sampleDescription}
\sampleIO{sample-1}
\sampleIO{sample-2}
\end{sampleDescription}

\end{document}
