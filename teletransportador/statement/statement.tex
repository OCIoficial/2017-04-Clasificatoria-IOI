\documentclass{oci}
\usepackage[utf8]{inputenc}
\usepackage{lipsum}

\title{Teletransportador cuántico de flujos}
\codename{teletransportador}

\begin{document}
\begin{problemDescription}
``Eureka!'' gritó el Dr. Bernardé al probar con éxito su último invento: el
teletransportador cuántico de flujos.
Con este invento, él planeaba revolucionar la industria de la transportación.

Pero este invento presento un gran problema para la gran nación de Nlogonia:
Dada su distribución consistente en una gran Metrópolis central con pequeñas
ciudades alrededor, cada una conectada por un camino a la capital, no hay forma
simple de organizar el costo del uso del transportador: Si el transportador
cuesta menos que el peaje del camino, este va a terminar colapsando por su mayor
uso; y en el caso contrario el camino colapsará.

Dada esta compleja situación, el presidente Robinçois llego a una sola
conclusión: el precio del teletransportador y el de uso de un camino deben ser
iguales.
Dado que el teletransportador es un servicio estatal, su coste debe ser
igual en todas las ciudades, así que se decidió que todos los caminos iban a ser
modificados de tal manera que todos tuvieran igual coste.

Esta no es una tarea fácil, pues cada camino tiene inicialmente un costo $A_i$ y
cambiar este costo en una unidad cuesta $C_i$ Nlogdolarés.
Es por esto que te han contratado a ti para responder la siguiente pregunta:
¿Cual es el coste $C$ óptimo para el teletransportador de tal manera que el costo
de cambiar los caminos sea mínimo?

\end{problemDescription}

\begin{inputDescription}
  Primero recibirás n ($2 \leq n \leq 10^6$): la cantidad de caminos.
  En la primera
linea recibiras n números: Los valores de $A_i$ .
Y en la última linea recibiras n números: 
Los valores de $C_i$ ($1 \leq C_i \leq 10^3$).
\end{inputDescription}

\begin{outputDescription}
Deberás imprimir un solo número: El valor para $C$ de tal manera que el costo de
cambiar todos los caminos sea mínimo.
En el caso de que existan múltiples $C$, imprime aquel con menor valor.
\end{outputDescription}

\begin{scoreDescription}
  \score{10} $A_i=A_j$ para todo $i,j$ y $1 \leq A_i \leq 10^9$.
  \score{20} $1 \leq A_i \leq 10$.
  \score{30} $C_i=C_j$ para todo $i,j$ y $1\leq A_i\leq 10^9$.
  \score{40} $1\leq A_i\leq 10^9$ y sin restricciones adicionales
\end{scoreDescription}

\begin{sampleDescription}
\sampleIO{sample-1}
\sampleIO{sample-2}
\end{sampleDescription}

\end{document}
