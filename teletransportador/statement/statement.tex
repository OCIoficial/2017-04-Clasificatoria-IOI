\documentclass{oci}
\usepackage[utf8]{inputenc}
\usepackage{lipsum}

\title{Teletransportador cuántico de flujos}
\codename{teletransportador}

\begin{document}
\begin{problemDescription}
  ``Eureka!'' gritó el Dr. Bernardé al terminar con éxito las pruebas de su
  último invento: el teletransportador cuántico de flujos.
  Con este invento, Bernadé planeaba revolucionar la industria de la
  teletransportación en la gran nación de Nlogonia.
  Sin embargo, a pesar de su innovadora tecnología este invento presentó un gran
  problema para las autoridades de Nlogonia quienes tenían que regular su uso
  antes de su construcción.

  La nación de Nlogonia tiene una distribución particular consistente en una gran
  Metrópolis central rodeada de varias ciudades pequeñas.
  Cada ciudad está conectada directamente con la Metrópolis central mediante
  una autopista y para utilizar las autopistas los nlogones deben pagar un peaje
  cuyo valor depende de las características del camino.

  Antes de su construcción, las autoridades de Nlogonia deben fijar el precio
  del uso del teletransportador.
  Esta no es una tarea sencilla, pues el precio que escojan puede tener severas
  consecuencias en la congestión de Nlogonia.
  Si el costo de uso del teletransportador es menor que el costo de uso de una
  autopista, este va a terminar colapsando por su mayor uso; y en el caso
  contrario la autopista colapsará.

  Después de un minucioso análisis en mano de sus expertos asesores, el
  presidente Robinçois determinó que la única solución posible es que el precio
  de uso del teletransportador y el de las autopistas sea el mismo.
  Dado que el teletransportador es un servicio estatal, su costo debe ser
  igual para todas las ciudades, así que la determinación es que todas las
  autopistas deben ser modificadas de manera que tengan el mismo costo de uso.

  En Nlogonia existen $N$ autopistas y cada una es identificada con un número
  entre 1 y $N$.
  Inicialmente la autopista $i$-ésima tiene un costo de uso de $A_i$ nlogdolares
  y modificarla para cambiar su costo de uso en una unidad cuesta $B_i$
  nlogdolares.
  Por ejemplo, si el costo inicial de uso de una autopista es 4 nlogdolares y
  el costo unitario de modificación es de 2 nlogdolares habría que pagar
  $(8-4)\times 2=4\times 2=10$ nlogdolares si se quiere modificar la autopista
  para que su costo de uso sea 8 nlogdolares.
  Similarmente habría que pagar $(4-1)\times 2=6$ nlogdolares para que el costo
  de uso de la autopista sea 1 nlogdolar.

  Dada esta difícil situación se ta ha encomendado responder la siguiente
  pregunta: ¿Cuál es el costo $C$ que debe tener el uso del teletransportador de
  tal manera que el costo total de modificar las autopistas sea mínimo?

\end{problemDescription}

\begin{inputDescription}
  La primera línea de la entrada contiene un único entero $N$ ($2 \leq N \leq
  10^6$) correspondiente a la cantidad de autopistas.
  La siguiente línea contiene $N$ enteros $A_1, A_2, \ldots, A_N$ describiendo el
  costo inicial de uso de las autopistas.
  La última línea contiene $N$ enteros $B_1, B_2, \ldots, B_N$ ($1 \leq B_i \leq
  10^3$) correspondientes al costo unitario de modificación de las autopistas.
\end{inputDescription}

\begin{outputDescription}
  La salida debe consistir en una única línea conteniendo un entero: el costo
  $C$ que debe tener el uso del teletransportador de forma que el costo total de
  modificar las autopistas sea mínimo.
  En el caso de que exista más de un valor $C$ con estas características, debes
  imprimir el menor de estos.
\end{outputDescription}

\begin{scoreDescription}
  \score{10} Se probarán varios casos donde $A_i=A_j$ para todo $i,j$ y además
  $1 \leq A_i \leq 10^9$.
  \score{23} Se probarán varios casos donde $1 \leq A_i \leq 10$.
  \score{32} Se probarán varios casos donde $N=2$ y $1\leq A \leq 10^9$.
  \score{45} Se probarán varios casos donde $1\leq A_i\leq 10^9$.
\end{scoreDescription}

\begin{sampleDescription}
  \sampleIO{sample-1}
  \sampleIO{sample-2}
  \sampleIO{sample-3}
  \sampleIO{sample-4}
\end{sampleDescription}

\end{document}
