\documentclass{oci}
\usepackage[utf8]{inputenc}
\usepackage{lipsum}

\title{Naruto vs Sasuki pelea completa (español latino)}
\codename{naruto}

\begin{document}
\begin{problemDescription}
En medio de la pelea más intensa que ha presenciado el mundo ninja, Sasuke le
menciona algo a su rival que lo hace retirarse rápidamente del lugar: ``Antes de
la pelea instalé $N$ bombas en la Aldea de la Hoja, en este momento deben estar
a punto de estallar''.

Naruto conoce bien a Sasuke y sabe que él domina tres elementos con los cuales
puede hacer explosivos: \emph{fuego}, \emph{electricidad} y \emph{sonido}.
Naruto sabe también que desactivar las bombas tomará tiempo, y para esto
recurrirá a una de las técnicas que más domina, los clones de sombra.

Después de haber encontrado y reunido las $N$ bombas, nuestro héroe las llevó a
un lugar seguro para desactivarlas.
Naruto deberá usar sus poderes de \emph{agua}, \emph{tierra} y \emph{vacío},
para neutralizar los explosivos y de esta forma desactivar las bombas.
Usando el poder de agua puede desactivar una bomba de fuego, usando el de
tierra una bomba de electricidad y finalmente usando el poder de vacío puede
desactivar una bomba de sonido.
Naruto debe desactivar las bombas lo más rápido posible y para esto utilizará la
técnica de clones que le permite hacer copias de si mismo y de esta forma poder 
desactivar varias bombas de forma simultánea.

Antes de utilizar la técnica de clones Naruto debe primero escoger un entero
positivo $k$.
Luego de realizar la técnica se crearán $k$ copias de todos los clones que
haya en ese momento.
Por ejemplo, inicialmente solo hay una copia de Naruto, si este decide realizar
la técnica de clones con $k=3$ habrá 3 copia de Naruto.
Si en este estado decide realizar la técnica con $k=2$ cada una de las 3 copias
se clonará 2 veces y por lo tanto habrá finalmente 6 copias de Naruto.

Controlar todos los clones simultáneamente es complejo y por lo tanto cada vez
que Naruto utiliza uno de sus poderes para desactivar una bomba todos los clones
deben usar el mismo poder y desactivar una bomba al mismo tiempo.
Adicionalmente no es posible eliminar clones mientras queden bombas por
desactivar.

Naruto puede utilizar la técnica de clones y multiplicarse de forma instantánea,
pero le toma un minuto usar uno de sus poderes para desactivar una bomba.
El tiempo apremia y nuestro ninja favorito quiere saber cuál es la mínima
cantidad de minutos en que puede desactivar todas las bombas.
Por ejemplo, si Sasuke ha instalado 7 bombas, repartidas en 3 de fuego, 2 de
electricidad y 2 de sonido, Naruto puede desactivar todas las bombas en 4
minutos siguiendo la siguiente secuencia de acciones.

\begin{enumerate}
\item Desactivar una bomba de fuego.
\item Clonarse usando $k = 2$. Ahora hay dos copias de Naruto.
\item Desactivar dos bombas de fuego simultáneamente.
\item Desactivar dos bombas de electricidad simultáneamente.
\item Desactivar dos bombas de sonido simultáneamente.
\end{enumerate}

\end{problemDescription}

\begin{inputDescription}
  La entrada consiste en una única línea conteniendo tres enteros $a, b$ y $c$
  ($0\leq a \leq b \leq c \leq 100$) correspondientes a la cantidad de bombas de fuego,
  electricidad y sonido respectivamente.
\end{inputDescription}

\begin{outputDescription}
  La salida debe contener un único entero correspondiente a la cantidad mínima
  de minutos en que Naruto puede desactivar todas las bombas.
\end{outputDescription}

\begin{scoreDescription}
  \score{8} Se probarán varios casos donde $a = b = 0$.
  \score{17} Se probarán varios casos donde $a = 0$ y $c - b \leq 2$.
  \score{32} Se probarán varios casos donde $c - a \leq 2$
  \score{43} Se probarán varios casos sin restricciones adicionales.
\end{scoreDescription}

\begin{sampleDescription}
\sampleIO{sample-1}
\sampleIO{sample-2}
\end{sampleDescription}

\end{document}
